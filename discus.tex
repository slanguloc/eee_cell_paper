\section{Discussion}
%% model
In this study we developed the most detailed computational model of mouse M1 microcircuits and corticospinal neurons up to date, based on an accumulated set of experimental studies. Overall, the model incorporates quantitative experimental data from 15 publications, thus integrating previously isolated knowledge into a unified framework. Unlike the real brain, this in silico system serves as a testbed that can be probed extensively and precisely, and can provide accurate measurements of neural dynamics at multiple scales. We employed the model to evaluate how long-range inputs from surrounding regions and molecular/pharmacological-level effects (e.g. regulation of HCN channel) modulate M1 dynamics. 

% main findings
We simulated the inputs from the main cortical and thalamic regions that project to M1, based on recent experimental data from optogenetic mapping studies. Consistent with experimental data \cite{Hook13,Sute15}, this study evidenced two different pathways that can generate corticospinal output: motor-related inputs from VL, cM1 or M2 regions bypassing upper M1 layers and directly projecting to corticospinal cells, and sensory-related inputs from PO, S1 and S2 regions projecting to superficial IT neurons in turn exciting layer 5B corticospinal cells. Activation of OC region lead to increased layer 6 IT and CT activity and physiological oscillations in the beta and gamma range. Downregulation of corticospinal HCN lead to increased firing rate, as observed experimentally. This has been hypothesized as a potential mechanism to translate action planning into action execution \cite{Shee11}.

% future 
Our study provides insights to help decipher the neural code underlying the brain circuits responsible for producing movement, and help understand motor disorders, including spinal cord injury. The computational model provides a framework to integrate experimental data and can be progressively extended as new data becomes available. This provides a useful tool for researchers in the field, who can use the framework to evaluate hypothesis and guide the design of new experiments. 


%% Limitations
\note{
- Still artifacts of binning in conn: eg. L5A (0.42-0.51) has very high IT2 input for bin 0.4375-0.5, but very low for prev (0.375-0.4375 -- mostly L4) and next (0.5-0.5626 -- mostly IT5B). 
- Could be resolved by fitting binned data to functions strength=f(yfrac), resulting in smoother conn profiles

%% M1 movement or absence of movement
\cite{Ebbe17} - Motor cortex — to act or not to act?
These data revealed that two-thirds of movement-related corticospinal neurons [in mouse M1] showed decreased activity during movement
\cite{Pete17} - Reorganization of corticospinal output during motor learning.
11\% - movement related
22\% - quiescent (not moving) related
12\% - always active
16\% - always silent
35\% - switches state from week 1 to week2

\cite{Chen17} - mouse M1 PTs also encode preparatory activity and object location (not just motor commands)

\cite{LiCh15} - "Pyramidal tract neurons are downstream of intratelencephalic neurons, which show contralateral and ipsilat- eral selectivity with little contralateral bias. This suggests that during movement planning distributed preparatory activity in intratelence- phalic neuron networks is converted into a movement command in pyramidal tract neurons (‘output-potent’ activity)39, which ultimately triggers directional movements."

\cite{Schie11} - dissociating motor cortex from motor -- activity without movement

\cite{Arle08}- Motor cortex stimulation for pain and movement disorders.

}
%% Interaction between PT cell and inhibitory populations
\note{
- Whisker behavior reduced activity in vM1 (can help justify decrease in activtiy in model)

- "Despite this promiscuous connectivity, some studies suggest that the PV population may preferentially inhibit
  specific PC subtypes (Ye et al., 2015), though reports in L5 conflict over whether IT or PT cells receive more inhibition
  (Fariñas and DeFelipe, 1991; Lee et al., 2014; Rock and Apicella, 2015).
  
- While PT cells receive and integrate input widely, they do not locally excite many other excitatory neurons, except for otherPT cells (Brown and Hestrin, 2009; Lefort et al., 2009; Harris and Shepherd, 2015; Jiang et al., 2015; Yamawaki and Shepherd, 2015). However, by harnessing the massive divergence of Martinotti (SOM) cells, L5 PT cells can potentially route inhibition to a large cohort of neurons across multiple layers. Burst firing by one or a small number of L5 PT cells might therefore represent a ‘‘call to order’’, quieting activity throughout an entire cortical column by activating frequency dependent disynaptic inhibition.



%% Future work
- Effect of dendritic distribution of synapses (sCRACM data) on circuit dynamics and neural coding (SfN16 poster).

- Study more in depth effect of H-current on spatial and temporal integration L2/3 inputs to PT (already have results reproducing figs 4, 7, 8, 9 and 11 from Sheets, 2011)

- Evaluate STR-SPI transformation hypothesis (rate vs temporal coding) via temporal-variability measures.

- Reproduce data in Dembrow, 2015: IT dendrites functioned as temporal integrators, particularly responsive to dendritic inputs within the gamma freq range (40 –140 Hz); PT dendrites acted as coincidence detectors, responding to spatially distributed signals in a narrow time window.

- Study role of interneurons: interlaminar disynaptic interactions (E->I->E); replicate ratios in Yamawaki,2015.

- Compare oscillations in different layers (eg. alpha=suppress -- use Lakatos paper, gamma=info flow, beta=resting?)

- Study correlations and synchrony: spike-triggered average; auto- and cross-correlations between spike trains; joint peri stimulus time histograms; cross-level coupling (CLC); spike train synchrony measures; Pairwise Phase consistency (PPC)

- Study effect of low vs high frequency synaptic inputs on network (the network analogous of cell synaptic resonance in apical dendrites for high frequencies)
}