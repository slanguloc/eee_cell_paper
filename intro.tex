\section*{Abstract}
%% Adapted from CNS17 and SFN17 abstracts
We developed a biologically detailed multiscale model of mouse primary motor cortex (M1) microcircuits, incorporating data from several recent experimental studies. The model simulates at scale a cylindrical volume with a diameter of 300 μm and cortical depth 1350 μm of M1. It includes over 10,000 cells distributed across cortical layers based on measured cell densities, with close to 30 million synaptic connections. Neuron models were optimized to reproduce electrophysiological properties of  major classes of M1 neurons. Layer 5 corticospinal and corticostriatal neuron morphologies with 700+ compartments reproduced cell 3D reconstructions, and their ionic channel distributions were optimized within experimental constraints to reproduce in vitro recordings. The network was driven by the main long-range inputs to M1: posterior nucleus (PO) and ventrolateral (VL) thalamus PO, primary and secondary somatosensory cortices (S1, S2), contralateral M1, secondary motor cortex (M2), and orbital cortex (OC). The network local and long-range connections depended on pre- and post-synaptic cell class and cortical depth. Data was based on optogenetic circuit mapping studies which determined that connection strengths vary within layer as a function of the neuron's cortical depth. The synaptic input distribution across cell dendritic trees -- likely to subserve important neural coding functions -- was also mapped using optogenetic methods and incorporated into the model. 
We employed the model to study the effect on M1 of increased activity from each of the long-range inputs, and of different levels of H-current in pyramidal tract-projecting (PT) corticospinal neurons. Microcircuit dynamics and information flow were quantified using firing rates, oscillations, and information transfer measures (Granger causality and normalized transfer entropy). Simulation results support the hypothesis that M1 operates along a continuum of modes characterized by the degree of activation of intratelencephalic (IT), corticothalamic (CT) and PT neurons. The particular subset of M1 neurons targeted by different long-range inputs and the level of H-current in PT neurons regulated the different modes. Downregulation of H-current facilitated synaptic integration of inputs and increased PT output activity. Therefore, VL, cM1 and M2 inputs with downregulated H-current promoted an PT-predominant mode; whereas PO, S1 and S2 inputs with upregulated H-current favored an IT/CT-predominant mode, but a mixed mode where upper layer IT cells in turn activated PT cells if H-current was downregulated. This evidenced two different pathways that can generate corticospinal output: motor-related inputs bypassing upper M1 layers and directly projecting to PT cells, and sensory-related inputs projecting to superficial IT neurons in turn exciting PT cells. These findings provide a hypothetical mechanism to translate action planning into action execution. Overall, our model serves as an extensible framework to combine experimental data at multiple scales and perform in silico experiments that help us understand M1 microcircuit dynamics, information flow and biophysical mechanisms, and potentially develop treatments for motor disorders.

\section{Introduction} 
%% Differentiation of M1 vs other cortical regions -- L4, L5B etc
Primary motor cortex (M1) microcircuits exhibit many differences with those of other cortices -- \eg\ visual \cite{Ship05} -- and with the traditional "canonical" microcircuit proposed by Douglas and Martin almost 30 years ago \cite{douglas89}. These include differences in the cytoarchitecture, cell classes and long and local connectivity. Notable characteristics that demarcate mouse M1 from sensory cortices are a narrower layer 4 and wider layer 5B \cite{Yama15}, and its strongly assymetrical projection: upper layer intratelencephalic (IT) cells connect to layer 5B pyramidal-tract (PT) cells but PT cells do not project back \cite{Weil08,Ande10}. Corticocortical and corticostriatal cells in superficial layers seem to play a role in recurrently processing information at a local level and within thalamocortical regions, whereas deep corticospinal cells act output information (motor commands) from M1 to spinal cord circuits \cite{Weil08,Ande10}. These distinctions have profound consequences in terms of understanding neural dynamics and information processing \cite{Shep13}, and should therefore be considered when developing detailed biophysical models of M1.

\note{
\cite{Rein10}-  "IT-type neurons conveying sensory and motor planning information to striatum and PT-type neurons conveying an efference copy of motor commands (for motor cortex at least)"

\cite{Cons13} -"Deep Cortical Layers are Activated Directly by Thalamus"; "thalamus activates two separate, independent “strata” of cortex in parallel."; "bistratified model"

\cite{Ship05} - differences of M1 vs V1 - "global conservation of some aspects of function, whereas regional variations in architecture can be used to chart the ‘organs’ of the cortex, and perhaps to understand their functional differences."

}
%% Role of corticostriatal (recursive superficial layers, integrate sensory info) vs corticospinal cells (output to spinal cord)
%Explore role of corticostriatal (recursive superficial layers, integrate sensory info) vs corticospinal cells (output to spinal cord).

%% Microcircuit model combining data at multiple scales from multiple studies on same species+regions (mouse M1)
We developed a detailed model of M1 microcircuits that combines data at multiple scales from multiple studies: including cell classes morphology, physiology, densities and proportions; and long-range, local, and dendritic-level connectivity. Most data was obtained from studies on the same species and region -- mouse M1 --  and combined into a single theoretical framework. To integrate all these experimental findings we developed several methodological novelties including specifying connections as a function of normalized cortical depth (NCD) instead of the standard layers, and incorporating projection-specific dendritic distributions of synaptic inputs extracted from subcellular Channelrhodopsin-2- Assisted Circuit Mapping (sCRACM) studies \cite{Hook13,Sute15}. Additionally, we developed a novel tool, NetPyNE, to help develop, simulate and analyze data-driven biological network models in NEURON. 

%% Explore multiscale effects - dendritic synaptic locations and Ih effect on network oscillations/info flow/dynamics
Our model, which contains over 10 thousand neurons, 30 miliion synapses and models a cylindric cortical volume with a depth of 1350 $\mu m$ and a diameter of 300 $\mu m$ constitutes the most biophysically detailed model of mouse M1 microcircuits available. It enables studying the non-intuitive dynamics and interactions occurring across multiple scales, with a level of resolution and precision which is not yet available experimentally. Cortical circuits can have extremely disparate responses to changes in their cellular and subcellular properties; detailed simulation provides a means to systematically explore the circuit-level consequences of manipulations of the cellular components and wiring of the system.

%% Multiple modes of operation; modulate those modes via input or HCN regulation
In this study we explore the effect on M1 of increased activity from the different long-range inputs projectig into M1; and of regulation of H-current in PT corticospinal cells. We evaluate the hypothesis that different that M1 operates along a continuum of modes characterized by the level of activation of intratelencephalic (IT), corticothalamic (CT) and PT neurons, and that long-range inputs and H-current modulates these which modes predominate. 

\note{channel on network oscillations / info flow.
Hypothesize/predict/validate multiple mode of operation (IT/CT vs CT) regulated by long-range inputs and H-current
}
