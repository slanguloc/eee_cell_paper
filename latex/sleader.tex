%% $Id: sleader.tex,v 1.8 2004/08/24 17:13:26 billl Exp $

\graphicspath{{id/}{figs/}{ps/}{pdf/}{gif/}{gif1/}{gif2/}{gif3/}{img/}}
\DeclareGraphicsExtensions{.ps,.id,.eps}

\usepackage[latin1]{inputenc}
\usepackage{pstricks,pst-node,pst-text,pst-3d}
\usepackage{amsmath}
% $Id: neuro.tex,v 1.21 2012/12/01 16:41:46 billl Exp $
% transmitters and receptors
\def\gabaa{GABA$_{\mathrm A}$}
\def\gabaaname{$\gamma$-amino butyric acid receptor A (\gabaa)}
\def\gabab{GABA$_{\mathrm B}$}
\def\gababname{$\gamma$-amino butyric acid receptor B (\gabab)}

% units and ions
\def\mum{$\mu$m}
\def\muM{$\mu$M}
\def\deg{^{\mathrm o}} %% ?? doesn't seem to be built in
\def\q10{Q$_{10}$}
\def\cm2{cm$^2$}
\def\uS{$\mu$S}
\def\us{$\mu$s}
\def\uScm2{$\mu S/cm^2$}
\def\mScm2{mS/cm$^2$}
\def\Scm2{S/cm$^2$}
\def\ohmcm{$\Omega \cdot$cm}
\def\ohmcm2{$\Omega$/cm$^2$}
\def\mufcm2{$\mu$F/cm$^2$}
\def\Mohm{M$\Omega$}
\def\kohm{k$\Omega$}
\def\erev{E$_{\mathrm{rev}}$}
\def\mg2{Mg$^{\mathrm{+2}}$}
\def\ca{Ca$^{2+}$}
\def\cl{Cl$^{-}$}
\def\cs{Cs$^{++}$}
\def\cd{Cd$^{2+}$}
\def\co{Co$^{2+}$}
\def\o2{O$_\mathrm{2}$}
\def\co2{CO$_\mathrm{2}$}
\def\ba{Ba$^{++}$}
\def\cai{[Ca$^{++}]_i$}
\def\mg{Mg$^{++}$}
\def\ni{Ni$^{+}$}
\def\na{Na$^+$}
\def\k{K$^+$}
\def\hco3{HCO$_3^-$}

% g's
\def\gmax{$\overline{\mathrm g}$}
\def\pmax{$\overline{\mathrm p}$}
\def\imax{$\overline{\mathrm i}$}
\def\gleak{$\overline{g}_{\mathrm{leak}}$}
\def\giT{$\overline{g}_{\mathrm{T}}$}
\def\gil{$\overline{g}_{\mathrm{L}}$}
\def\gim{$\overline{g}_{\mathrm{M}}$}
\def\gic{$\overline{g}_{\mathrm{C}}$}
\def\gin{$\overline{g}_{\mathrm{N}}$}
\def\gir{$\overline{g}_{\mathrm{R}}$}
\def\gh{$g_{\mathrm{h}}$}
\def\gmaxh{$\overline{g}_{\mathrm{h}}$}
\def\ggabaa{$\overline{g}_{\mathrm{GABA A}}$}
\def\ggabab{$\overline{g}_{\mathrm{GABA B}}$}
\def\gnmda{$\overline{g}_{\mathrm{NMDA}}$}
\def\gampa{$\overline{g}_{\mathrm{AMPA}}$}
\def\giahp{$\overline{g}_{\mathrm{AHP}}$}
\def\gim{$\overline{g}_{\mathrm M}$}
\def\gkca{$\overline{g}_{\mathrm C}$}
\def\gT{$\overline{g}_{\mathrm T}$}
\def\glnmda{GLU$_{\mathrm{NMDA}}$}
\def\glampa{GLU$_{\mathrm{AMPA}}$}
\def\glmetab{GLU$_{\mathrm{METAB}}$}
\def\gk{$\overline{g}_{\mathrm{K}}$}
\def\gna{$\overline{g}_{\mathrm{Na}}$}
\def\gglu{$\overline{g}_{\mathrm{GLU}}$}
\def\ggaba{$\overline{g}_{\mathrm{GABA}}$}

% i's
\def\iahp{$I_{\mathrm{AHP}}$}
\def\ic{$I_{\mathrm C}$}
\def\il{$I_{\mathrm L}$}
\def\ileak{$I_{\mathrm{leak}}$}
\def\iT{$I_{\mathrm T}$}
\def\ikk{$I_{\mathrm{K}}$}
\def\ik2{$I_{\mathrm{K2}}$}
\def\ikdr{$I_{\mathrm{Kdr}}$}
\def\ia{$I_{\mathrm A}$}
\def\id{$I_{\mathrm D}$}
\def\im{$I_{\mathrm M}$}
\def\ina{$I_{\mathrm{Na}}$}
\def\inaf{$I_{\mathrm{Naf}}$}
\def\inap{$I_{\mathrm{Nap}}$}
\def\ip3{IP$_{\mathrm{3}}$}  %%%% not sure why these are failing \ip is defined somehow?
\def\ip3r{IP$_{\mathrm{3}}$R}
\def\ipass{$I_{\mathrm{pass}}$}
\def\ih{$I_{\mathrm h}$}
\def\icat{$I_{\mathrm T}$}
\def\ican{$I_{\mathrm{CAN}}$}
\def\ica{$I_{\mathrm{Ca}}$}
\def\ical{$I_{\mathrm L}$}
\def\ikca{$I_{\mathrm C}$}
\def\igabaa{$I_{\mathrm{GABA A}}$}
\def\igabab{$I_{\mathrm{GABA B}}$}
\def\iglnmda{$I_{\mathrm{NMDA}}$}
\def\iglampa{$I_{\mathrm{AMPA}}$}
\def\iglmetab{$I_{\mathrm{METAB}}$}

% from pfc paper
\def\dcaapp{$\mathrm{\bf D}_{\mathrm{Ca}(App)}$}
\def\pipp{\ensuremath{\mathrm{PtdIns(4,5)P}_2}}
\def\ippp{\ensuremath{\mathrm{IP_3}}}
\def\IPPPR{\ensuremath{\mathrm{IP3R}}}
\def\ipppr{\ensuremath{\mathrm{IP_3R}}}
\def\jserca{\ensuremath{J_{\mathrm{SERCA}}}}
\def\jleak{\ensuremath{J_{\mathrm{leak}}}}
\def\jip{\ensuremath{J_{\mathrm{IP3R}}}}
\def\CA{$Ca^{2+}$}
\def\IPThree{$\mathrm{IP_3}$}
\def\cac{\mathrm{Ca_{cyt}^{2+}}}
\def\cion{\ensuremath{J_\mathrm{memb}}}
\def\cae{\mathrm{Ca_{ER}^{2+}}}
\def\IPThreeR{$\mathrm{IP_3R}$}
\def\Ihsm{cAMP}
\def\camp{cAMP}
\def\Ihm{$g_h$}
\def\Ihg{$g_{h}$}
\def\dcab{$\Delta CaB$}
\def\PC{pyramidal cell}
\def\BC{basket cell}
\def\qten{$Q_{10}$}

% other terms
\def\pvalxy{P$_{\mathrm{xy}}$}
\def\pvalee{P$_{\mathrm{ee}}$}
\def\pvalae{P$_{\mathrm{ae}}$}
\def\pburst{P$_{\mathrm{burst}}$}
\def\semburst{SEM$_{\mathrm{burst}}$}
\def\mpburst{$\overline{P}_{\mathrm{burst}}$}
\def\eleak{E$_{\mathrm{leak}}$}
\def\ena{E$_{\mathrm{Na}}$}
\def\ek{E$_{\mathrm{K}}$}
\def\Ek{E$_{\mathrm{K}}$}
\def\esyn{E$_{\mathrm{syn}}$}
\def\eca{E$_{\mathrm{Ca}}$}
\def\Egabaa{E$_{\mathrm{GABA A}}$}
\def\Vhalf{V$_{\mathrm{\frac{1}{2}}}$}
\def\VhT{\Vhalf$_{\mathrm{T-channel}}$}
\def\vmemb{V$_\mathrm{memb}$}
\def\rin{R$_{in}$}
% \def\rm{R$_m$}
\def\cm{C$_m$}

% Hodgkin-Huxley-isms
\def\minf{$m_\infty$}
\def\minfv{$m_\infty(V)$}
\def\hinf{$h_\infty$}
\def\ninf{$n_\infty$}
\def\taum{$\tau_m$}
\def\tauh{$\tau_h$}
\def\taun{$\tau_n$}

% names of things
\def\ANN{artificial neural network}
\def\BG{basal ganglia}
\def\hh{Hodgkin-Huxley}
\def\HH{Hodgkin-Huxley}
\def\ach{acetylcholine}
\def\lgn{lateral geniculate nucleus}
\def\Ahp{after hyperpolarizing currents}
\def\lts{low-threshold spike}
\def\tc{thalamocortical}
\def\re{reticularis}
\def\tcn{TC neuron}
\def\ren{RE neuron}
\def\tcns{TC neurons}
\def\rens{RE neurons}
\def\nrt{nucleus reticularis thalami}
\def\trn{thalamic reticular nucleus}
\def\bzd{benzodiazepine}
\def\Inviv{{\em In vivo}}
\def\Invit{{\em In vitro}}
\def\inviv{{\em in vivo}}
\def\invit{{\em in vitro}}
\def\Insil{{\em In silico}}
\def\insil{{\em in silico}}
\def\ttx{tetrodotoxin}
\def\htcc{high-threshold calcium channel}
\def\ltcc{low-threshold calcium channel}
\def\kca{\ca-sensitive \k\ channel}
\def\RNN{realistic neural network}

\def\sbc{subiculum}
\def\ecx{entorhinal cortex}
\def\pp{perforant path}
\def\ib{intrinsic bursting}
\def\regf{regular firing}
\def\iyd{in your dreams}

% hippocampus
\def\hipp{hippocampus}
\def\DG{dentate gyrus}
\def\EC{entorhinal cortex}
\def\HF{hippocampal formation}

% thalamus
\def\TIs{thalamic interneurons}
\def\TI{thalamic interneuron}
\def\SU{single unit}
\def\PYR{cortical pyramidal cell}
\def\TC{thalamocortical cell}
\def\RE{reticularis cell}
\def\TCs{thalamocortical cells}
\def\REs{reticularis cells}
\def\ISI{interspike interval}
\def\IBI{interburst interval}
\def\IAB{integrate-and-burst model}
\def\IAF{integrate-and-fire model}
\def\wagrij{WAG/r$_{ij}$}

% math stuff
\def\dvdt{\ensuremath\frac{\mathrm{d}V}{\mathrm{d}t}}
\def\dlvdlt{\ensuremath\frac{\Delta V}{\Delta t}}
\def\dxdt{\ensuremath\frac{\mathrm{d}x}{\mathrm{d}t}}
\def\dxxdtt{\ensuremath{\frac{\mathrm{d}^2\!x}{\mathrm{d}t^2}}} % dxdt2 doesn't work


%%% Local Variables: 
%%% mode: latex
%%% TeX-master: t
%%% End: 

% $Id: misc.tex,v 1.47 2013/01/09 19:40:59 billl Exp $

% \usepackage{ifthen}
% \usepackage{mycite}
% \usepackage{amsmath}

\ifx\graphicspath\undefined
  \ifx\pdfoutput\undefined
    \usepackage[dvips]{graphicx}
    \DeclareGraphicsExtensions{.ps,.id,.eps}
  \else
    \usepackage[pdftex]{graphicx}
    \DeclareGraphicsExtensions{.pdf}
    \usepackage{type1cm}
  \fi
  \graphicspath{{id/}{figs/}{ps/}{pdf/}{gif/}{img/}{gif2/}}
\else
  \message{**** include misc before graphicx? ****} %%% or use \message
\fi

% convenient functions
\def\bpath{/usr/site/config/papers} %%%% site of our .bib files

% scientific notation (sn) 
\newcommand{\sn}[2]{$\mathrm #1 \cdot 10^{#2} $}
% \Exp1.1(4)
\def\Exp#1(#2){\ensuremath{#1 \cdot 10^{#2}}}
% \e1.1e4
\def\e#1e#2{{#1} $\!\;\cdot\!$ 10$^{#2}$}
% \p1.1e2,4
\def\p#1e#2,#3{\ensuremath{\mathrm #1 \cdot #2^{#3}}}

\newcommand{\code}[1]{\\ {\tt #1} \\ \noindent}
\newcommand{\nibf}[1]{\noindent{\bf #1}}
\newcommand{\niem}[1]{\noindent{\em #1}}
\def\ra{$\rightarrow$}
\def\etal{{\em et al.}}
\def\cf{{\em cf.}}
\def\ie{{\em i.e.,}}
\def\Ie{{\em I.e.,}}
\def\eg{{\em e.g.,}}
\def\etc{{\em etc}}
\def\Eg{{\em E.g.,}}
\newcounter{c2} 
\newcounter{cnt}
\newcounter{fig}
\setcounter{fig}{0}
\setcounter{c2}{0}
\def\inc0{\setcounter{c2}{0}}
\def\incn{\noindent \stepcounter{c2}{1}\arabic{c2}. }

\def\sstretch{1.0}
\newcommand{\newstretch}[1]{\setstretch{#1} \small\normalsize}
\newcommand{\restretch}[0]{\setstretch{\sstretch} \small\normalsize}
\newcommand{\float}[1]{\parbox[htb][#1in][0in]{6in}{~~~}

}

\def\bpm{\begin{pmatrix}}    \def\epm{\end{pmatrix}}
\def\beq{\begin{equation}}   \def\eeq{\end{equation}}
\def\be{\begin{equation*}}   \def\ee{\end{equation*}}
\def\bip{\begin{inparaenum}[\bfseries{}1.]} \def\eip{\end{inparaenum}} 
\def\bipp{\begin{inparaenum}[\itshape{}a.]} %%%% for nested listing 

\newcommand{\bfig}[3]{
\begin{figure}[htb]
\begin{center}
\epsfig{figure=#1,height=#2in,width=#3in}
\end{center}
}
\newcommand{\efig}[0]{\end{figure}}

\newcommand{\bfigg}[2]{
\begin{figure}[htb]
\begin{center}
\epsfig{figure=#1,height=#2in}
\end{center}
}

% formatting
\ifx\note\undefined
  \newcommand{\note}[1]{} % Create a way of putting in in-text notes
\else
  \renewcommand{\note}[1]{} % need to redefine \note{} anyway in case someone redefined it with diff args
\fi
\newcommand{\BAD}[1]{#1} % if using the wrong word identify the fact with BAD
\newcommand{\NOTE}[1]{} % more visible note
\newcommand{\notnote}[1]{{#1}} % something in process of going back and forth
\newcommand{\Note}[1]{\hspace*{2in}{\em \bf $\langle\langle$ Note: #1 $\rangle\rangle$ }} % show a note
\newcommand{\mydef}[2]{\global\def#1{#2}{#2}} % see also \suba in papers/grant/m1u01/ctl.tex

% begin and end numbered list
\newcommand{\blist}[0]{\begin{list}{\arabic{cnt}.}{\usecounter{cnt}}}
\newcommand{\elist}[0]{\end{list}}
\def\bbar{\cbstart}
\def\ebar{\cbend}

% figures
\newcommand{\pref}[1]{(see p. \pageref{#1})}
\newcommand{\cref}[1]{Chap.~\ref{#1}}
\newcommand{\eref}[1]{Eqn.~\ref{#1}}
\newcommand{\fref}[1]{Fig.~\ref{#1}}
\newcommand{\bfref}[1]{\noindent {\bf Figure~\ref{#1}}}
\newcommand{\figref}[1]{Figure~\ref{#1}}
\newcommand{\newfig}[1]{Fig.~\ref{#1}}
\newcommand{\newfigure}[1]{Figure~\ref{#1}}
\newcommand{\figlegend}[2]{Fig.~\ref{#1}: {#2}}

%% \showfig{file}{label}{[ABC]} 
%% eg \showfig{figs/d94apr26.05.id}{2cell}{} 
\newcommand{\showfig}[2]{
   \pagebreak
   \begin{figure}[htb]
   \includegraphics[scale=1,clip]{#1} 
   {\noindent \bf Fig.~\ref{#2}}
   \end{figure}
}
%% \showbmp{file}{label}{[ABC]} 
%% same as showfig but needed for bitmaps so don't stretch them
\newcommand{\showbmp}[3]{
   \pagebreak
   \begin{figure}[htb]
   \centerline{Figure~\ref{#2}#3}
   %%%% \epsfig{figure=#1}
   \end{figure}
}
%% show the top of 2 or more figures on 1 page
\newcommand{\showfiga}[3]{
   \pagebreak
   \begin{figure}[htb]
   \epsfig{figure=#1,height=#2in,width=#3in}
   \end{figure}
}
%% show the last of 2 or more figs on the page
\newcommand{\showfigb}[4]{
   \begin{figure}[htb]
   \centerline{Figure~\ref{#2}
   \epsfig{figure=#1,height=#3in,width=#4in}}
   \end{figure}
}
%% \showfigc{file}{label}{caption} 
%% eg \showfigc{figs/d94apr26.05.id}{2cell}{The caption} 
\newcommand{\showfigc}[3]{
   \pagebreak
   \begin{figure}[htb]
   \caption{#3}
   \centerline{Figure~\ref{#2}
   \epsfig{figure=#1,height=9in,width=7in}}
   \end{figure}
}

%% \figl{file}{label}{caption} 
%% eg \figl{figs/d94apr26.05.id}{2cell}{The caption} 
\newcommand{\figl}[3]{
   \begin{figure}[htb]
   \begin{center}
   \epsfig{figure=#1,height=3.5in}
   \caption{#3}\label{#2}
   \end{center}
   \end{figure}
}

%% redo for use with graphicx
%% \figi{label}{caption} -- file is in figs/label.id
%% eg \figi{2cell}{The caption} 
\newcommand{\figi}[2]{
  \begin{figure}
    \refstepcounter{figure}
    \label{#1}
    \begin{center}
      \includegraphics[scale=.5,clip]{#1} 
    \end{center}
    {\noindent \bf Fig.~\ref{#1}: #2} 
  \end{figure}
}

%% \figj{label}{caption} -- file is in figs/label.jpg
\newcommand{\figj}[2]{
  \begin{figure}
    \refstepcounter{figure}
    \label{#1}
    \begin{center}
      \includegraphics[scale=.5,clip]{#1.jpg} 
    \end{center}
    {\noindent \bf Fig.~\ref{#1}: #2} 
  \end{figure}
}

%% rotate
%% \figr{label}{caption} -- file is in figs/label.id
%% eg \figr{2cell}{The caption} 
\newcommand{\figr}[2]{
   \refstepcounter{figure}
   \begin{figure}[htb]
   %% \begin{figure}
   \begin{center}
   \label{#1}
   \includegraphics[scale=.5,angle=270,clip]{#1} 
   \end{center}
   % \caption{#2}
   {\noindent \bf Fig.~\ref{#1}: #2}
   \end{figure}
}

%% allow choice of scaling
\newcommand{\figrs}[3]{
%%   \begin{figure}[htb]
   \begin{figure}
   \refstepcounter{figure}
   \begin{center}
   \label{#1}
   \includegraphics[scale=#2,angle=270,clip]{#1} 
   \end{center}
   % \caption{#3}
   {\noindent \bf Fig.~\ref{#1}: #3}
   \end{figure}
}

%% scaling + rotation
\newcommand{\figs}[3]{
%%   \begin{figure}[htb]
   \begin{figure}
   \refstepcounter{figure}
   \label{#1}
   \begin{center}
   \includegraphics[scale=#2,clip]{#1} 
   \end{center}
   % \caption{#3}
   {\noindent \bf Fig.~\ref{#1}: #3}
   \end{figure}
}

% figos -- without the begin{figure} end{figure}
\newcommand{\figos}[3]{
   \begin{center}
   \label{#1}
   \includegraphics[scale=#2,clip]{#1} 
   \end{center}
   {\noindent \bf Fig.~\ref{#1}: #3}
}

% ig -- for making slides
\newcommand{\ig}[2]{
   \begin{center}
   \includegraphics[scale=#2]{#1} 
   \end{center}
}

% igr -- rotate ig
\newcommand{\igr}[2]{
   \begin{center}
     \vspace*{-0.2in}
   \includegraphics[scale=#2,angle=270]{#1} 
   \end{center}
}

% testing
\newcommand{\been}[0]{\begin{enumerate}}
\newcommand{\enen}[0]{\end{enumerate}}
\newcommand{\choices}[5]{\begin{enumerate} \item{#1} \item{#2} \item{#3} \item{#4} \item{#5} \end{enumerate}}

%%% Local Variables: 
%%% mode: latex
%%% TeX-master: t
%%% End: 


%% Definition of new colors
\newrgbcolor{LemonChiffon}{1. 0.98 0.8}
\newrgbcolor{LightBlue}{0.68 0.85 0.9}

%% > BEGIN OF OVERLAPPED COLORS
%% Code below devised by Denis Girou (CNRS/IDRIS - France, Denis.Girou@idris.fr)
\newrgbcolor{LemonChiffon}{1. 0.98 0.8}
\newrgbcolor{LightBlue}{0.68 0.85 0.9}
\makeatletter
\newdimen\pst@dimz

%% Draw two overlapped surfaces, with computation of the mixed color for
%% the intersection of the surfaces 
%% #1=first  surface, #2=color of first  surface,
%% #3=second surface, #4=color of second surface
\def\ColoredOverlappedSurfaces#1#2#3#4{%%
\psset{fillstyle=solid}
%% Decode the three components of the first RGB color
\DecodeRGBFirstColor{\csname color@#2\endcsname}%%
\psset{fillcolor=#2}
%% Draw first surface
#1
%% Decode the three components of the second RGB color
\DecodeRGBSecondColor{\csname color@#4\endcsname}%%
%% Compute the mixed color
\BuildMixedColor
%% Draw second surface
\psclip{\psset{fillcolor=#4}#3}
\psset{fillcolor=MixedColor}
%% Redraw overlapped surface in the mixed color
#1
\endpsclip}

%% Get the three components of the first color
\def\DecodeRGBFirstColor#1{%%
\pst@expandafter\pst@getnumiii{#1} {} {} {} {}\@nil
\edef\pst@FirstColorR{\pst@tempg}%%
\edef\pst@FirstColorG{\pst@temph}%%
\edef\pst@FirstColorB{\pst@tempi}%%
%%\typeout{Color 1=\pst@tempg,\pst@temph,\pst@tempi}%% Debug
}

%% Get the three components of the second color
\def\DecodeRGBSecondColor#1{%%
\pst@expandafter\pst@getnumiii{#1} {} {} {} {}\@nil
\edef\pst@SecondColorR{\pst@tempg}%%
\edef\pst@SecondColorG{\pst@temph}%%
\edef\pst@SecondColorB{\pst@tempi}%%
%%\typeout{Color 2=\pst@tempg,\pst@temph,\pst@tempi}%% Debug
}

%% Build the mixed RBG color (by means of each three components)
\def\BuildMixedColor{%%
%% Resulting R component
\pst@dimz=\pst@FirstColorR pt
\advance\pst@dimz\pst@SecondColorR pt
\divide\pst@dimz\tw@
\pst@dimtonum{\pst@dimz}{\pst@MixedColorR}%%
%% Resulting G component
\pst@dimz=\pst@FirstColorG pt
\advance\pst@dimz\pst@SecondColorG pt
\divide\pst@dimz\tw@
\pst@dimtonum{\pst@dimz}{\pst@MixedColorG}%%
%% Resulting B component
\pst@dimz=\pst@FirstColorB pt
\advance\pst@dimz\pst@SecondColorB pt
\divide\pst@dimz\tw@
\pst@dimtonum{\pst@dimz}{\pst@MixedColorB}%%
%% Definition of the mixed color MixedColor
\newrgbcolor{MixedColor}{%%
\pst@MixedColorR\space \pst@MixedColorG\space \pst@MixedColorB}
%%\typeout{Mixed color=\csname color@MixedColor\endcsname}%% Debug
}
\makeatother
%% < END OF OVERLAPPED COLORS

