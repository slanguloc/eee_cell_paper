%% $Id: sleader.tex,v 1.8 2004/08/24 17:13:26 billl Exp $

\graphicspath{{id/}{figs/}{ps/}{pdf/}{gif/}{gif1/}{gif2/}{gif3/}{img/}}
\DeclareGraphicsExtensions{.ps,.id,.eps}

\usepackage[latin1]{inputenc}
\usepackage{pstricks,pst-node,pst-text,pst-3d}
\usepackage{amsmath}
\input{neuro}
\input{misc}

%% Definition of new colors
\newrgbcolor{LemonChiffon}{1. 0.98 0.8}
\newrgbcolor{LightBlue}{0.68 0.85 0.9}

%% > BEGIN OF OVERLAPPED COLORS
%% Code below devised by Denis Girou (CNRS/IDRIS - France, Denis.Girou@idris.fr)
\newrgbcolor{LemonChiffon}{1. 0.98 0.8}
\newrgbcolor{LightBlue}{0.68 0.85 0.9}
\makeatletter
\newdimen\pst@dimz

%% Draw two overlapped surfaces, with computation of the mixed color for
%% the intersection of the surfaces 
%% #1=first  surface, #2=color of first  surface,
%% #3=second surface, #4=color of second surface
\def\ColoredOverlappedSurfaces#1#2#3#4{%%
\psset{fillstyle=solid}
%% Decode the three components of the first RGB color
\DecodeRGBFirstColor{\csname color@#2\endcsname}%%
\psset{fillcolor=#2}
%% Draw first surface
#1
%% Decode the three components of the second RGB color
\DecodeRGBSecondColor{\csname color@#4\endcsname}%%
%% Compute the mixed color
\BuildMixedColor
%% Draw second surface
\psclip{\psset{fillcolor=#4}#3}
\psset{fillcolor=MixedColor}
%% Redraw overlapped surface in the mixed color
#1
\endpsclip}

%% Get the three components of the first color
\def\DecodeRGBFirstColor#1{%%
\pst@expandafter\pst@getnumiii{#1} {} {} {} {}\@nil
\edef\pst@FirstColorR{\pst@tempg}%%
\edef\pst@FirstColorG{\pst@temph}%%
\edef\pst@FirstColorB{\pst@tempi}%%
%%\typeout{Color 1=\pst@tempg,\pst@temph,\pst@tempi}%% Debug
}

%% Get the three components of the second color
\def\DecodeRGBSecondColor#1{%%
\pst@expandafter\pst@getnumiii{#1} {} {} {} {}\@nil
\edef\pst@SecondColorR{\pst@tempg}%%
\edef\pst@SecondColorG{\pst@temph}%%
\edef\pst@SecondColorB{\pst@tempi}%%
%%\typeout{Color 2=\pst@tempg,\pst@temph,\pst@tempi}%% Debug
}

%% Build the mixed RBG color (by means of each three components)
\def\BuildMixedColor{%%
%% Resulting R component
\pst@dimz=\pst@FirstColorR pt
\advance\pst@dimz\pst@SecondColorR pt
\divide\pst@dimz\tw@
\pst@dimtonum{\pst@dimz}{\pst@MixedColorR}%%
%% Resulting G component
\pst@dimz=\pst@FirstColorG pt
\advance\pst@dimz\pst@SecondColorG pt
\divide\pst@dimz\tw@
\pst@dimtonum{\pst@dimz}{\pst@MixedColorG}%%
%% Resulting B component
\pst@dimz=\pst@FirstColorB pt
\advance\pst@dimz\pst@SecondColorB pt
\divide\pst@dimz\tw@
\pst@dimtonum{\pst@dimz}{\pst@MixedColorB}%%
%% Definition of the mixed color MixedColor
\newrgbcolor{MixedColor}{%%
\pst@MixedColorR\space \pst@MixedColorG\space \pst@MixedColorB}
%%\typeout{Mixed color=\csname color@MixedColor\endcsname}%% Debug
}
\makeatother
%% < END OF OVERLAPPED COLORS

