% figures
\newcommand{\cref}[1]{Chap.~\ref{#1}}
\newcommand{\eref}[1]{Eqn.~\ref{#1}}
\newcommand{\fref}[1]{Fig.~\ref{#1}}
\newcommand{\bfref}[1]{\noindent {\bf Figure~\ref{#1}}}
\newcommand{\figref}[1]{Figure~\ref{#1}}
\newcommand{\newfig}[1]{\refstepcounter{figure}\label{#1}Fig.~\ref{#1}}
\newcommand{\newfigure}[1]{\refstepcounter{figure}\label{#1}Figure~\ref{#1}}

%% \showfig{file}{label}{[ABC]} 
%% eg \showfig{figs/d94apr26.05.id}{2cell}{} 
\newcommand{\showfig}[2]{
   \pagebreak
   \begin{figure}[htb]
   \includegraphics[scale=1,clip]{#1} 
   {\noindent \bf Fig.~\ref{#2}}
   \end{figure}
}
%% \showbmp{file}{label}{[ABC]} 
%% same as showfig but needed for bitmaps so don't stretch them
\newcommand{\showbmp}[3]{
   \pagebreak
   \begin{figure}[htb]
   \centerline{Figure~\ref{#2}#3}
   %%%% \epsfig{figure=#1}}
   \end{figure}
}
%% show the top of 2 or more figures on 1 page
\newcommand{\showfiga}[3]{
   \pagebreak
   \begin{figure}[htb]
   \epsfig{figure=#1,height=#2in,width=#3in}
   \end{figure}
}
%% show the last of 2 or more figs on the page
\newcommand{\showfigb}[4]{
   \begin{figure}[htb]
   \centerline{Figure~\ref{#2}
   \epsfig{figure=#1,height=#3in,width=#4in}}
   \end{figure}
}
%% \showfigc{file}{label}{caption} 
%% eg \showfigc{figs/d94apr26.05.id}{2cell}{The caption} 
\newcommand{\showfigc}[3]{
   \pagebreak
   \begin{figure}[htb]
   \caption{#3}
   \centerline{Figure~\ref{#2}
   \epsfig{figure=#1,height=9in,width=7in}}
   \end{figure}
}

%% \figl{file}{label}{caption} 
%% eg \figl{figs/d94apr26.05.id}{2cell}{The caption} 
\newcommand{\figl}[3]{
   \begin{figure}[htb]
   \begin{center}
   \epsfig{figure=#1,height=3.5in}
   \caption{#3}\label{#2}
   \end{center}
   \end{figure}
}

%% redo for use with graphicx
%% \figi{label}{caption} -- file is in figs/label.id
%% eg \figi{2cell}{The caption} 
\newcommand{\figi}[2]{
  \begin{figure}
    \refstepcounter{figure}
    \label{#1}
    \begin{center}
      \includegraphics[scale=.5,clip]{#1} 
    \end{center}
    {\noindent \bf Fig.~\ref{#1}: #2} 
  \end{figure}
}

%% rotate
%% \figr{label}{caption} -- file is in figs/label.id
%% eg \figr{2cell}{The caption} 
\newcommand{\figr}[2]{
   \refstepcounter{figure}
%%   \begin{figure}[htb]
   \begin{figure}
   \begin{center}
   \label{#1}
   \includegraphics[scale=.5,angle=270,clip]{#1} 
   \end{center}
   % \caption{#2}
   {\noindent \bf Fig.~\ref{#1}: #2}
   \end{figure}
}

%% allow choice of scaling
\newcommand{\figrs}[3]{
%%   \begin{figure}[htb]
   \begin{figure}
   \refstepcounter{figure}
   \begin{center}
   \label{#1}
   \includegraphics[scale=#2,angle=270,clip]{#1} 
   \end{center}
   % \caption{#3}
   {\noindent \bf Fig.~\ref{#1}: #3}
   \end{figure}
}

%% scaling + rotation
\newcommand{\figs}[3]{
%%   \begin{figure}[htb]
   \begin{figure}
   \refstepcounter{figure}
   \label{#1}
   \begin{center}
   \includegraphics[scale=#2,clip]{#1} 
   \end{center}
   % \caption{#3}
   {\noindent \bf Fig.~\ref{#1}: #3}
   \end{figure}
}

% figos -- without the begin{figure} end{figure}
\newcommand{\figos}[3]{
   \begin{center}
   \label{#1}
   \includegraphics[scale=#2,clip]{#1} 
   \end{center}
   {\noindent \bf Fig.~\ref{#1}: #3}
}


%%% Local Variables: 
%%% mode: latex
%%% TeX-master: t
%%% End: 
